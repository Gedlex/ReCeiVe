%%%%%%%%%%%%%%%%%%%%%%%%%%%%%%%%%%%%%%%%%
% ReCeiVe
% LuaLaTeX Template
% Version 1.12.0 (10/10/2020)
%
% Original authors:
% Ged Lex (gedlex@hotmail.ch)
%
% Important note:
% This template must be compiled with LuaLaTeX, the below lines will ensure this
%!TEX TS-program = lualatex
%!TEX encoding = UTF-8 Unicode
%%%%%%%%%%%%%%%%%%%%%%%%%%%%%%%%%%%%%%%%%

%----------------------------------------------------------------------------------------
%	PACKAGES AND OTHER DOCUMENT CONFIGURATIONS
%----------------------------------------------------------------------------------------
\documentclass[rightPos]{ReCeiVe}      % By default, use 'letterpaper' for US letter
%\geometry{letterpaper, % paper size
%          left=1.75cm, right=1.75cm, top=1.5cm, bottom=1.5cm, footskip=.6cm, headsep=.5cm, % margins
%          showframe}                  % show geometry frame
%          heightrounded}              % avoid underfull vbox warning
          
% Color for highlights
%\definecolor{highlight}{HTML}{EB9534} % Specify your own color
%\colorlet{highlight}{white}           % Set predefined color
% Default colors include: darkgray, gray, lightgray, lightblue, orange, red, concrete

% Colors for text - uncomment and modify
%\definecolor{darktext}{HTML}{414141}
%\definecolor{text}{HTML}{414141}
%\definecolor{graytext}{HTML}{414141}
%\definecolor{lighttext}{HTML}{414141}

%----------------------------------------------------------------------------------------
%	PERSONAL INFORMATION
%	Comment any of the lines below if they are not required
%----------------------------------------------------------------------------------------

% Usage: \photo[circle|rectangle,edge|noedge,fill|nofill,left|right]{size,<path-to-image>}
\newcommand{\specialcell}[2][l]{%
	\renewcommand{\arraystretch}{.5}
	\begin{tabular}[c]{@{}#1@{}}#2\end{tabular}
}
\photo[circle,edge,fill,sidebar]{2.5cm}{pics/pp2.pdf}
\name{Ged}{Lex}
\address{\specialcell{My very long, long, long adress with\\ street name, number postcode, country}}
\mobile{+01 23 456 78 90}
\email{gedlex@hotmail.ch}
\github{gedlex}
%\linkedin{linkedin}
%\homepage{www.homepage.com}
%\twitter{@twit}
%\xing{xing name}
%\stackoverflow{SOid}{SOname}
%\skype{skypeid}
%\reddit{reddit account}
%\extrainfo{info}

\background{pics/background5.pdf}
\position{BSc in Mechanical Engineering} % Your expertise/fields
\headwords{Engineer, software geek, technical enthusiast and carpenter} % A few headwords to describe yourself
\quote{"Some men see things as they are, and ask why. I dream of things that never were, and ask why not." - Robert Kennedy} % A quote or statement

%----------------------------------------------------------------------------------------
%	SIDEBAR CONTENT
%	Fill in the information you would like to add to your sidebar
%----------------------------------------------------------------------------------------

\aboutMe{Engineer, software geek, technical enthusiast and carpenter}
\skillset[label/0\hfill +500 h]{MATLAB/.8,Python/.4,VBA/.6,Java/.2,C++/.4}
\languages{{English/Fluent (C1 Certificate)},{German/Mother tongue},{French/Conversational}}
\hobbies{Lorem ipsum dolor sit amet, consectetur adipiscing elit}
\nonProfit{%
	\begin{itemize}
	\item{Lorem ipsum dolor sit amet}
	\item{Lorem ipsum dolor sit amet, consectetur adipiscing elit. Donec finibus eu felis ullamcorper finibus.}
	\end{itemize}
}
% Usage: \userSection[<section title>]{<content>}
% \userSection[title]{content}

%----------------------------------------------------------------------------------------
\begin{document}
% Print the header
% Usage: \makecvheader[<position>]
\makecvheader[L]

% Print the sidebar
% Usage: \makecvsidebar[xOffset/value,yOffset/value,noRadius|radius]
\makecvsidebar

%----------------------------------------------------------------------------------------
% Usage: \makecvfooter(<left>}{<center>}{<right>)
\makecvfooter{\today}{Ged Lex~~~·~~~CV}{\thepage}

%----------------------------------------------------------------------------------------
%	CV/RESUME CONTENT
%	Each section is imported separately, open each file in order to modify it
%----------------------------------------------------------------------------------------
% Remove space before first section
\vspace{-\acvSectionTopSkip}
\section{Education}
\cventry
{BSc in Mechanical Engineering} % Education Title
{School of Engineering} % Organization
{Turicum, CH} % Location
{09/2017 - 08/2020} % Date
\begin{cvitems}
\item {Major in Systems Engineering and Automation and in Lightweight design}
\item {Bachelor Thesis: Comparison of Deformation and Lifetime Prediction Models for Steam Turbine Valves}
\item {Project Thesis: Control Design of a Stepper Motor}
\end{cvitems}

\cventry
{Professional Maturity} % Education title
{BBW} % Organization
{Turicum, CH} % Location
{08/2012 - 08/2015} % Date

\cventry
{Licensed Carpenter, Federal Diploma of Vocational Education and Training} % Job title
{Gewerbliche Berufsschule} % Organization
{Turicum, CH} % Location
{08/2012 - 08/2015} % Date
\section{Professional Experience}
\cventry
{Internship as Test Engineer} % Job Title
{Technology Company AG} % Organization
{Turicum, CH} % Location
{11/2020 - today} % Date
\begin{cvitems}
\item {Testing and QA}
\item {Failure Analysis}
\item {Lifetime prediction}
\item {Software technical implementation}
\end{cvitems}

\cventry
{Internship as Mechanical Manufacturer} % Job Title
{Mechanical Manufacturer AG} % Organization
{Turicum, CH} % Location
{07/2019 - 09/2019} % Date
\begin{cvitems}
\item {Conventional turning and milling}
\item {Electropolishing and finishing}
\end{cvitems}

\cventry
{Licensed Carpenter} % Job Title
{Timber Work GmbH} % Organization
{Turicum, CH} % Location
{07/2018 - 09/2018} % Date
\begin{cvitems}
\item {General timber construction works}
\end{cvitems}

\cventry
{Licensed Carpenter} % Job Title
{Timber Work GmbH} % Organization
{Turicum, CH} % Location
{08/2012 - 08/2017} % Date
\begin{cvitems}
\item {General timber construction works}
\item {Tutoring of an apprentice}
\end{cvitems}
\section{Honors}
\begin{cvhonors}

\cvhonor
{1st Place} % Position
{Circular Economy Challenge} % Event
{Turicum, CH} % Location
{2021} % Date

\end{cvhonors}

%----------------------------------------------------------------------------------------
%	SECOND PAGE
%----------------------------------------------------------------------------------------
\pagebreak
% Usage: \photo[circle|rectangle,edge|noedge,fill|nofill,left|right]{size,<path>}
%\photo[rectangle,edge,fill,sidebar]{2.5cm}{pics/pp2.pdf}
\aboutMe{Something else}
\skillset[label/0\hfill +500 h]{Ansys/.3,Abaqus/.5,Catia/.6}
\languages{{Russian/Non},{Italian/Non}}
\hobbies{\LaTeX{} or somewhat}
\nonProfit{%
	\begin{itemize}
	\item{Hello}
	\item{World}
	\end{itemize}
	}
% Usage: \makecvheader[<position>]
%\makecvheader[C]

% Usage: \makecvsidebar[xOffset/value,yOffset/value,noRadius|radius]
%\makecvsidebar

\end{document}